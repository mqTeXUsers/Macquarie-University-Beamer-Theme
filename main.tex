% Latex template: mahmoud.s.fahmy@students.kasralainy.edu.eg
% For more details: https://www.sharelatex.com/learn/Beamer

% Slide Masters:

% Title
% Text
% 2 column
% Full-image
% Bibliography
% Closing
\PassOptionsToPackage{table}{xcolor}    % https://tex.stackexchange.com/a/5365/5483
\documentclass[aspectratio=169, 12pt]{beamer} % Aspect ratio
% https://tex.stackexchange.com/a/14339/5483 
% Possible values: 1610, 169, 149, 54, 43 and 32.
% 169 = 16:9

\usetheme{macquarie}

\usepackage[english]{babel}       % Set language
\usepackage[utf8x]{inputenc}      % Set encoding
\usepackage{colortbl}
\mode<presentation>           % Set options
{
  \usetheme{default}          % Set theme
  \usecolortheme{default}         % Set colors
  \usefonttheme{default}          % Set font theme
  \setbeamertemplate{caption}[numbered] % Set caption to be numbered
}

% Uncomment this to have the outline at the beginning of each section highlighted.
%\AtBeginSection[]
%{
%  \begin{frame}{Outline}
%    \tableofcontents[currentsection]
%  \end{frame}
%}

\usepackage{graphicx}         % For including figures
\usepackage{booktabs}         % For table rules
\usepackage{hyperref}         % For cross-referencing


\setbeamertemplate{footline}[frame number] 

\title{Title for a minimal beamer presentation} % Presentation title
\author{Author One}               % Presentation author
\institute{Name of department/affiliation}         % Author affiliation
\date{\today\\
Here is some extra text to test length of line. Maybe presentation url here?}                 % Today's date  
\begin{document}

% Title page
% This page includes the informations defined earlier including title, author/s, affiliation/s and the date
% \begin{frame}[noframenumbering]

\maketitle

  
% \end{frame}

% Outline
% This page includes the outline (Table of content) of the presentation. All sections and subsections will appear in the outline by default.
\begin{frame}{Outline}
  \tableofcontents
\end{frame}

% The following is the most frequently used slide types in beamer
% The slide structure is as follows:
%
%\begin{frame}{<slide-title>}
% <content>
%\end{frame}

\section{Section One\\Where the section title keeps going and going and going and going.}

\begin{frame}{Slide with bullet points}
  This is a bullet list of two points:
    \begin{itemize}
    \item Point one
        \item Point two
  \end{itemize}
\end{frame}

\begin{frame}{Slide with two columns}
  \begin{columns}
    \column{.5\textwidth}
        Text goes in first column.
        
        \column{.5\textwidth}
        Text goes in second column
  \end{columns}
\end{frame}

\section{Section Two}

\begin{frame}{Slide with table}
  \input{tables/table1.tex}
\end{frame}

\begin{frame}{Slide with different table}
  \input{tables/table2.tex}
\end{frame}

\begin{frame}{Slide with figure}
  \begin{figure}[H]
    \centering
        \includegraphics[height=.7\textheight]{branding/MQ_INT_VER_RGB_POS.pdf}
        \caption{Caption for figure one. MQ logo vertical.}
        \label{fig:figure1}
  \end{figure}
\end{frame}

\begin{frame}{Slide with references}
  This is to reference a figure (Figure \ref{fig:figure1})\\
    This it to reference a table (Table \ref{tab:table1})\\
    This is to cite an article \cite{Sobotkova2015-lq, Nosek2018-yv}\\
    This is to add an article to the references without mentioning in the text \nocite{Borgman2015-rh, Center_for_Open_Science2018-pn}\\
\end{frame}
\section{References}

% Adding the option 'allowframebreaks' allows the contents of the slide to be expanded in more than one slide.
% The "1" comes from the outer theme"
\begin{frame}[allowframebreaks]{References}
  \tiny
  \bibliography{references}
  \bibliographystyle{apalike}
\end{frame}


\begin{frame}[standout]
Thank you!
\end{frame}



\end{document}
